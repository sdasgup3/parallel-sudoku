\documentclass[10pt,twoside]{article}
\usepackage{graphicx}
\usepackage{url}
\usepackage{hyperref}
\hypersetup{
	colorlinks=true,
	citecolor=blue,%
    filecolor=blue,%
    linkcolor=blue,%
    urlcolor=blue
}

\newcommand{\doctitle}{%
Graph Coloring}
\newcommand{\cmnt}[1]{}

\pagestyle{myheadings}
\markboth{\hfill\doctitle}{\doctitle\hfill}

\bibliographystyle{siam}

\addtolength{\textwidth}{1.00in}
\addtolength{\textheight}{1.00in}
\addtolength{\evensidemargin}{-1.00in}
\addtolength{\oddsidemargin}{-0.00in}
\addtolength{\topmargin}{-.50in}

\hyphenation{in-de-pen-dent}

%\title{\textbf{\doctitle}\\
\title{\textbf{Graph Coloring Using State Space Search}}

\author{Sandeep Dasgupta\thanks{Electronic address: \texttt{sdasgup3@illinois.edu}}
\qquad Tanmay Gangwani\thanks{Electronic address: \texttt{gangwan2@illinois.edu}}
\qquad Mengjia Yan\thanks{Electronic address: \texttt{mengjia@illinois.edu}}
\qquad  Novi Singh \thanks{Electronic address: \texttt{novi.singh94@gmail.com}}} 

\begin{document}

\thispagestyle{empty}

\maketitle
\section{Heuristics for State-Space Search}
The following are the heuristics that we intend to implement in our parallel algorithm. Some of the ideas are borrowed from \cite{Kale1995}.
\begin{itemize}  \item Pre-Coloring and Vertex Removal 
    \begin{itemize} 
      \item Coloring a clique : By searching for a clique of a given size and
        coloring it arbitrarily we can reduce the number of states that need to
        be searched as the algorithm do not have to color the vertices involved
        in a clique in different part of the state space search.  We would
        consider a clique size of 3 since larger cliques can be formed from
        combining size-3 cliques.

    \item Removal of vertices with fewer neighbors than k, the number of colors:
      Any vertex fewer than k neighbors is guaranteed to have a color available
      to it. These are removed at the starting chare and placed in a local
      stack for latter LIFO merging. Also, at every stage of the state space
      search, a local algorithm tries to find if the number of colors available
      to a vertex (Total number of colors - number of different colors used to
          color the neighboring colored vertices ) exceeds the number of its
      uncolored vertices.  If that is the case, that vertex is removed since it
      is guaranteed to have a color available to it, and is stored locally.
    \end{itemize}

  \item Detection of Independent Subgraphs     \\
    This is a simple heuristic which allow us to color smaller graphs
    independently. However, in the case where a graph is connected we can find
    an articulation point (a vertex that once removed would split the graph
        into two independent subcomponents), color that vertex, and then color
    the subgraphs. This heuristic can be applied recursively to obtain any
    number of independent graphs that can be colored and then composed. This
    kind of a splitting generates an AND-OR space search tree instead of just
    an OR tree.  The child chares that are coloring the subgraph maintain
    proxies to their parents. If a child finds a solution then it reports to
    its parents and the message continues up the tree. If a solution is not
    found then the child will report up to its parent and the important point
    is that the parent needs to recursively kill all of the child chares that
    it spawned. This is the important distinction about the AND-OR tree, namely
    that a failure at an AND node (where we split the graph at an articulation
        point) leads to failure for all child nodes of the AND node.
    Furthermore, to speed up killing child chares, all chares wishing to spawn
    a child must ask their parent whether or not they can spawn a child, thus
    reducing the potential depth a kill command might have to travel down to
    complete.

  \item Impossibility Testing and Forced Moves
    \begin{itemize}
      \item Impossibility testing: If we find that by coloring a vertex that
        one of its neighbors is reduced to 0 possible colors then we don’t
        generate that state. This helps to reduce total chare count.  
        
      \item Forced Move: If by coloring something we find that one of its neighbors is now
        forced to be a certain color because it only has one color available
        then we perform that move. However, this might then lead to another
        vertex not having any possible colors left, which means that the forced
        move heuristic has to be recursive in nature.
    \end{itemize}      

  \item Value Ordering and Prioritization: When a chare selects the most
    constraint vertex to color, and spawns off ‘m’ child chares, where ‘m’ is
    the number of possible colors for the vertex, then we assign bitvector
    priorities to the child chares. The chare in which the assigned color least
    impacts the color possibilities of the neighboring vertices is given the
    highest priority. In this way, we sweep the state space search in a left to
    right manner.

  \item Adaptive grain size control for Chare creation: By grain size we mean
  the number of vertices colored at each state of the state space search tree.  
  If each state selects 1 vertex (most constraint) and spawns $k$ children
  ($k$ = color possibilities for that vertex), then the
  number of spawned chares grows as: 
  \[ 
  1 + k + k^2 + k^3 + k^4  + \dots + k^d  = O(K^{d+1} - 1). 
  \] where $d$ is depth of the state space tree.
  
  Note that, this is not an exact estimate since the branching factor would reduce
  (be much less than $k$ as we go down the state space tree since the coloring
   possibilities reduce). Let the grain size be $G$. This means that each node
  would select the $g$ most constraint vertices and then spawn $k^G$ children. Total
  chares spawned in this case is: 
  Let $a = \frac{d}{G}$
  \[
  1 + k^G + k^{2G} + k^{3G} + \dots + k^{aG} =  O(k^{aG + 1 - G}) = O (K ^ {d + 1 - G})	
  \]
  Comparing the two, we get a reduction in number of chares spawned by a factor of $k^G$ for a grain size of
  G.  
  A more aggressive way of controlling grain size, which would lead to a less
  optimal solution, but with much less number of chares: 
      At each level of state space, a chare colors (permanent coloring) G
      vertices, then finds the next most constraint vertex, and spawns off $k$
      children. The permanent coloring of $G$ vertices could be done using some
      sequential graph-coloring algorithm.  The number of chares in this case
      grows as: 
      Let  m=d/G, d = original depth of state space tree, G = grain size
      \[
      1 + k + k^2 + k^3 + \dots + k^m = O (k^{\frac{d}{G}}) 
      \]
      
      Here, we get a reduction of a factor of $k^{d-\frac{d}{G}}$ for a grain size of $G$.
              This is much more than the previous reduction of $k^G$, and hence
              we would find a solution much faster, using much less memory. But
              the downside is that we will not get the most optimal coloring.
\end{itemize}  


\section[Code Status]{Code Status\footnote{Find the .ci interface file at \url{https://github.com/sdasgup3/ParallelSudoku/blob/master/Source/graphColor.ci} }}
\begin{itemize}

  \item Input graph: We have a python interface to the main chare. A random
    graph generated by python is fed to the main chare and stored as an
    adjacency list.  
    
  \item Conservative Estimate: A linear time sequential algorithm
    is run on the input graph to conservatively estimate the chromatic number
    of the graph. This algorithm simply assigns the lowest possible color to a
    newly discovered vertex in the graph.  
    
  \item Parallel graph coloring:  Starting from the conservative estimate, say
  X, we run repeatedly the state space search algorithm with chromatic numbers
  X, X-1, X-2 and so on. If Y is the chromatic number in the first iteration
  not to yield a full coloring solution to the graph, then Y+1 is the optimal
  coloring. This is reported and verified by the main chare.  In the present
  implementation, a child chare reports its findings back to the parent. The
  results are percolated all the way to the main chare, which then declares if
  the input graph is colorable with the chromatic number selected for the
  running iteration. We use an SDAG for the parent to wait on its children, and
  marshaled parameters for data transfer. The heuristics mentioned in the
  separate file are not currently used to prune the search space and optimize
  for fast results. We plan to work on these next.  
    
  \item  State Space Evolution: The state space search is akin to a tree. A
    node in this tree represents the whole input graph with some partial
    coloring, and is handled by a singleton chare. The input graph, stored as
    an adjacency list, is made a read-only data structure. A new chare receives
    the partial coloring of the graph from its parent, runs local algorithms on
    it, and then passes on the new partial coloring to its children in the
    tree. At a very high level, the local algorithms perform the task of
    selecting the next few vertices in the graph to color, assigning them
    colors, and firing off new chares. The algorithms are guided by various
    heuristics as discussed above.

\end{itemize}

\section{References}

\nocite{*}
\bibliography{CS598_project_proposal}

\end{document}
